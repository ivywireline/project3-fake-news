\documentclass[10pt,letterpaper]{article}
\usepackage[latin1]{inputenc}
\usepackage{amsmath}
\usepackage{amsfonts}
\usepackage{amssymb}
\usepackage{graphicx}
\usepackage[margin=1in]{geometry}
\usepackage{listings}
\usepackage{float}
\usepackage{fancyhdr}
\pagestyle{fancy}
\lhead{\today}
\chead{Project 2}
\rhead{Tan, Zhou}
%\usepackage[margin=1in]{geometry}
\usepackage{color} %red, green, blue, yellow, cyan, magenta, black, white
\definecolor{mygreen}{RGB}{28,172,0} % color values Red, Green, Blue
\definecolor{mylilas}{RGB}{170,55,241}

\newcommand{\ssbracket}[2]{#1^{(#2)}}

\author{Hao Hui Tan(999741711, tanstev1)\\Kyle Zhou (1000959732, zhoukyle)}
\title{CSC411H1S Project 3}
\begin{document}
	\lstset{language=Python,%
		%basicstyle=\color{red},
		breaklines=true,%
		%morekeywords={matlab2tikz},
		keywordstyle=\color{blue},%
		morekeywords=[2]{1}, keywordstyle=[2]{\color{black}},
		identifierstyle=\color{black},%
		stringstyle=\color{mylilas},
		commentstyle=\color{mygreen},%
		showstringspaces=false,%without this there will be a symbol in the places where there is a space
		numbers=left,%
		numberstyle={\tiny \color{black}},% size of the numbers
		numbersep=9pt, % this defines how far the numbers are from the text
		emph=[1]{for,end,break},emphstyle=[1]\color{red}, %some words to emphasise
		%emph=[2]{word1,word2}, emphstyle=[2]{style},
		caption=\lstname,
	}
	
	\maketitle
	\newpage
	\begin{enumerate}
		\item %1
		The Real headline data set seems to be larger than the Fake headline data set.
		Most of the headlines for both data sets are in English, but there are some French and Spanish headlines, as well as possibly other languages.
		
		Fake headlines seem to use ``Trump'' to refer to Donald Trump, while real headlines tend to use ``Donald Trump.''
		Fake headlines also tend to use more sensational or inflammatory terms such as declaring something ``an hilarious fail,'' or have grammatical mistakes like the aforementioned example.
		Headlines in general are all lowercase with no punctuation.
		However, it seems that the real headlines tend to be truncated, while the fake headlines seem to all have the full text.
		Some of the headlines are also misspelled (e.g. ``x jinpingi'' instead of ``xi jinping'').
		
		It is difficult to categorize headlines solely based on keywords, since the same word in different contexts could be either sensational, or factual.
		Some useful keywords could be ``racist'' (5 occurrences in fake, 2 occurrences in real), ``hillary,'' (18 occurrences in real, 97 occurrences in fake), and ``rigged'' (3 occurrences in real, 15 occurrences in fake).
		\item %2
		\item %3
		\begin{enumerate}
			\item %3a
			\item %3b
			\item %3c
		\end{enumerate}
	\item %4
	\item %5
	\item %6
	\begin{enumerate}
		\item %6a
		\item %6b
		\item %6c
	\end{enumerate}
	\item %7
	\begin{enumerate}
		\item %7a
		\item %7b
		\item %7c
	\end{enumerate}
	\item %8
	\begin{enumerate}
		\item %8a
		\item %8b
	\end{enumerate}
	\end{enumerate}
\end{document}